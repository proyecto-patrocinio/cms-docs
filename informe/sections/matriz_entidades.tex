\chapter{Matriz de Relaciones}\label{cap:apendix-matrix-relation}

En esta sección, se presenta la matriz de relaciones, una herramienta utilizada para el diseño de la base de datos del sistema. La matriz de relaciones permite visualizar y comprender las interconexiones entre las diversas entidades que componen la estructura de la base de datos.

El objetivo fundamental de este análisis es identificar y descubrir las relaciones entre las entidades, proporcionando una visión clara de cómo interactúan y se relacionan unos con otros.

A lo largo de la matriz, se representan las relaciones entre entidades específicas, asignándoles significado y nombres a estas conexiones. Cada celda de la matriz detalla cómo las instancias de una entidad están relacionadas con las instancias de otra entidad, y se proporciona una descripción contextual que ayuda a comprender la naturaleza de la conexión entre ellas.

La información contenida en la matriz de relaciones se deriva de un proceso de modelado y refinamiento continuo hasta alcanzar la tercera forma normal, un estándar en la normalización de bases de datos que promueve la eficiencia y evita redundancias.


\begin{sidewaystable}[p]
    \centering
        \begin{tabular}{|p{4.1cm}|*{6}{>{\raggedright\arraybackslash}p{2.5cm}|}}
         \hline
         & \textbf{Calendar} & \textbf{Event} & \textbf{Card} & \textbf{Panel} & \textbf{Board} & \textbf{Request} \textbf{Consultation} \\
         \hline
        \textbf{Calendar} & \cellcolor{gray!25} & Contiene & Contenido en & x & x & x \\
         \hline
        \textbf{Event} & Contenido en & \cellcolor{gray!25} & x & x & x & x \\
         \hline
        \textbf{Card} & Tiene un & x & \cellcolor{gray!25} & Contenida en & x & x  \\
         \hline
        \textbf{Panel} & x & x & Contiene & \cellcolor{gray!25} & Contenido en & x \\
         \hline
        \textbf{Board} & x & x & x & Tiene & \cellcolor{gray!25} &   Le llegan\\
         \hline
        \textbf{RequestConsultation} & x & x & x & x & Hacia un & \cellcolor{gray!25} \\
         \hline
        \textbf{Consultation} & x & x & Tiene una & x & x & Tiene \\
         \hline
        \textbf{Client} & x & x & x & x & x & x  \\
         \hline
        \textbf{Tel} & x & x & x & x & x & x \\
         \hline
        \textbf{User} & x & x & x & x & Miembro de & x \\
         \hline
        \textbf{Comment} & x & x & Está en una & x & x & x \\
         \hline
        \textbf{File} & x & x & x & x & x & x  \\
         \hline
        \textbf{Locality} & x & x & x & x & x & x  \\
         \hline
        \textbf{Province} & x & x & x & x & x & x  \\
         \hline
        \textbf{Nationality} & x & x & x & x & x & x  \\
         \hline
        \textbf{Child} & x & x & x & x & x & x  \\
         \hline
        \end{tabular})
        \caption{Matriz de Relaciones Primera Parte}
        \label{mat:der}
        \end{sidewaystable}


        
\begin{sidewaystable}[p]
    \centering
        \begin{tabular}{|p{4.1cm}|*{7}{>{\raggedright\arraybackslash}p{2.5cm}|}}
         \hline
         & \textbf{Consultation} & \textbf{Client} & \textbf{Tel} & \textbf{User} & \textbf{Comment} & \textbf{File} \\
         \hline
        \textbf{Calendar} & x & x & x & x & x & x \\
         \hline
        \textbf{Event} & x & x & x & x & x & x \\
         \hline
        \textbf{Card}  & Pertenece a & x & x & x & x & x \\
         \hline
        \textbf{Panel}  & x & x & x & x & x & x \\
         \hline
        \textbf{Board}  & x & x & x & Tiene Miembros & x & x \\
         \hline
        \textbf{RequestConsultation} &  Solicita enviar una & x & x & x & x & x \\
         \hline
        \textbf{Consultation} & \cellcolor{gray!25} & Realizada por & x & x & Tiene & x \\
         \hline
        \textbf{Client}  & Realiza una & \cellcolor{gray!25} & Tiene & x & x & x \\
         \hline
        \textbf{Tel}  & x & Pertenece a &\cellcolor{gray!25} & x & &   \\
         \hline
        \textbf{User} & x & x & x & \cellcolor{gray!25} & Realiza un & x \\
         \hline
        \textbf{Comment}  & x & x & x & Lo comenta & \cellcolor{gray!25} & Tiene \\
         \hline
        \textbf{File}  & x& x& x & x & Adjuntado en & \cellcolor{gray!25}\\
         \hline
        \textbf{Locality} & x & Residida por & x & x & x & x \\
         \hline
        \textbf{Province} & x & x & x & x & x & x \\
         \hline
        \textbf{Nationality} & x & x & x & x & x & x \\
         \hline
        \textbf{Child}  & x & Es hijo de & x & x & x & x \\
         \hline
        \end{tabular})
        \caption{Matriz de Relaciones Segunda Parte}
        \label{mat:der2}
        \end{sidewaystable}


        
\begin{sidewaystable}[p]
    \centering
        \begin{tabular}{|p{4.1cm}|*{5}{>{\raggedright\arraybackslash}p{2.5cm}|}}
         \hline
       &\textbf{Locality} & \textbf{Province} & \textbf{Nationality} & \textbf{Child} \\
         \hline
        \textbf{Calendar} & x & x & x & x \\
         \hline
        \textbf{Event} & x & x & x & x \\
         \hline
        \textbf{Card}  & x & x & x & x \\
         \hline
        \textbf{Panel}  & x & x & x & x \\
         \hline
        \textbf{Board}  & x & x & x & x \\
         \hline
        \textbf{RequestConsultation} & x & x & x & x \\
         \hline
        \textbf{Consultation}  & x & x & x & x \\
         \hline
        \textbf{Client} & Reside en & x & x & Tiene \\
         \hline
        \textbf{Patrimony}  & x & x & x & x \\
         \hline
        \textbf{Tel}   & x & x & x & \\
         \hline
        \textbf{User}  & x & x & x & x  \\
         \hline
        \textbf{Comment}   & x & x & x & x \\
         \hline
        \textbf{File} & x & x & x & x \\
         \hline
        \textbf{Locality}  & \cellcolor{gray!25} & Está en & x & Residida por \\
         \hline
        \textbf{Province}  & Tiene & \cellcolor{gray!25} & Está en & x \\
         \hline
        \textbf{Nationality}& x & Tiene & \cellcolor{gray!25} & x \\
         \hline
        \textbf{Child}  & Reside en & x & x & \cellcolor{gray!25} \\
         \hline
        \end{tabular})
        \caption{Matriz de Relaciones Tercera Parte}
        \label{mat:der3}
        \end{sidewaystable}
