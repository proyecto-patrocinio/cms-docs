\chapter{Diccionario de Base de Datos}\label{cap:dict-db}

Este capítulo proporciona una descripción detallada de las tablas de la base de datos y sus respectivos atributos. Es fundamental destacar que, para mantener la coherencia con el código interno, los valores de las variables de los atributos que contienen opciones se han establecido siguiendo el estándar del código y utilizando variables internas en inglés. Es importante tener en cuenta que estas variables no serán accesibles para el usuario final. En su lugar, la interfaz gráfica será responsable de interpretar y presentar valores más amigables que representen cada opción de manera comprensible para el usuario final.

\section{Tabla Board}\label{sec:table-board}
Un \texttt{board} representa el espacio de trabajo de una comisión en forma de tablero. Este tablero contiene paneles para agrupar tarjetas de consultas y organizar los casos y las solicitudes para el ingreso de nuevos casos a la comisión.

\begin{table}[htbp]
\centering
\label{tab:board}

\begin{tabular}{|p{3cm}|p{2.5cm}|p{4.5cm}|p{4cm}|p{1cm}|}
\hline
\textbf{Atributo} & \textbf{Tipo de dato}  & \textbf{Descripción} & \textbf{Comentarios} & \textbf{No nulo} \\ \hline
id & int4 & Clave Primaria. & Incremental automático. & V \\ \hline
title & varchar(45) & Titulo o nombre de la comisión. &  & V \\ \hline
\end{tabular}
\caption{Tabla Board}
\end{table}



\section{Tabla Board-User}\label{sec:table-board-user}
La tabla \texttt{Board-User} Registra la asociación entre usuarios y tableros, lo que permite gestionar los permisos y el acceso de cada usuario a los recursos del tablero correspondiente. Se utiliza para asignar usuarios a sus respectivas comisiones.


\begin{table}[htbp]
\centering
\label{tab:board-user}

\begin{tabular}{|p{3cm}|p{2.5cm}|p{4.5cm}|p{4cm}|p{1cm}|}
\hline
\textbf{Atributo} & \textbf{Tipo de dato}  & \textbf{Descripción} & \textbf{Comentarios} & \textbf{No nulo} \\ \hline
id & int4 & Clave Primaria. & Incremental automático. & V \\ \hline
created\_at & timestamptz & Fecha de asignación. &  & V \\ \hline
board\_id & int4 & PK del tablero. &  & V \\ \hline
user\_id & int4 & PK del usuario. &  & V \\ \hline
\end{tabular}
\caption{Tabla Board-User}
\end{table}



\section{Tabla Calendar}\label{sec:table-calendar}
La tabla \texttt{Calendar} representa el calendario asociado a un ticket de consulta. Se creó una tabla aparte para permitir escalabilidad y la posibilidad de agregar campos adicionales en el futuro, como la fecha de última modificación o la capacidad de crear múltiples calendarios para grupos de trabajo en una consulta.

\begin{table}[htbp]
\centering
\label{tab:calendar}

\begin{tabular}{|p{3cm}|p{2.5cm}|p{4.5cm}|p{4cm}|p{1cm}|}
\hline
\textbf{Atributo} & \textbf{Tipo de dato}  & \textbf{Descripción} & \textbf{Comentarios} & \textbf{No nulo} \\ \hline
id & int4 & Clave Primaria. & Incremental automático. & V \\ \hline
card\_id & int4 & PK de la tarjeta o ticket de consulta. &  & V \\ \hline
\end{tabular}
\caption{Tabla Calendar}
\end{table}



\section{Tabla Event}\label{sec:table-calendar}
La tabla \texttt{Event} representa el evento en un calendario.

\begin{table}[htbp]
\centering
\label{tab:event}

\begin{tabular}{|p{3cm}|p{2.5cm}|p{4.5cm}|p{4cm}|p{1cm}|}
\hline
\textbf{Atributo} & \textbf{Tipo de dato}  & \textbf{Descripción} & \textbf{Comentarios} & \textbf{No nulo} \\ \hline
id & int4 & Clave Primaria. & Incremental automático. & V \\ \hline
title & varchar(45) & Título del evento. &  & V \\ \hline
description & text & Descripción del evento. &  & F \\ \hline
start & timestamptz & Fecha de inicio. &  & V \\ \hline
end & timestamptz & Fecha de cierre. &  & V \\ \hline
calendar\_id & int4 & PK del calendario. &  & V \\ \hline
\end{tabular}
\caption{Tabla Event}
\end{table}


\section{Tabla Card}\label{sec:table-calendar}
Una \texttt{Card} se utiliza para ubicar una consulta dentro de un tablero de organización representada por una tarjeta la cual será colocada dentro de un panel.

\begin{table}[htbp]
\centering
\label{tab:card}

\begin{tabular}{|p{3cm}|p{2.5cm}|p{4.5cm}|p{4cm}|p{1cm}|}
\hline
\textbf{Atributo} & \textbf{Tipo de dato}  & \textbf{Descripción} & \textbf{Comentarios} & \textbf{No nulo} \\ \hline
consultation\_id & int4 & Clave primaria y PK de la consulta asociada. &  & V \\ \hline
tag & varchar(256) & Etiqueta o título del ticket. &  & V \\ \hline
panel\_id & int4 & PK del panel donde se encuentra ubicado el ticket.. &  & V \\ \hline
\end{tabular}
\caption{Tabla Card}
\end{table}



\section{Tabla Child}\label{sec:table-child}
La tabla \texttt{Child} almacena los datos de un hijo de un cliente.

\begin{table}[htbp]
\centering
\label{tab:child}

\begin{tabular}{|p{3cm}|p{2.5cm}|p{4.5cm}|p{4cm}|p{1cm}|}
\hline
\textbf{Atributo} & \textbf{Tipo de dato}  & \textbf{Descripción} & \textbf{Comentarios} & \textbf{No nulo} \\ \hline
id & int4 & Clave Primaria. & Incremental automático. & V \\ \hline
first\_name & varchar(150) & Nombre. &  & V \\ \hline
last\_name & varchar(150) & Apellido. &  & V \\ \hline
id\_type & varchar(10) & Tipo de documento. & Valores posibles: DOCUMENT y PASSPORT. & V \\ \hline
id\_value & varchar(20) & Valor del documento (Pasaporte o Número de documento). & & V \\ \hline
sex & varchar(6) & Sexo. & Valores posibles: MALE y FEMALE. & V \\ \hline
birth\_date & date & Fecha de nacimiento. & & V \\ \hline
address & text & Dirección donde vive. & & V \\ \hline
locality\_id & int4 & PK de localidad donde vive. & & V \\ \hline
family\_client- \_user\_id & int4 & PK de la familia asociada. & & V \\ \hline
\end{tabular}
\caption{Tabla Child}
\end{table}




\section{Tabla Client}\label{sec:table-client}
Un cliente en el contexto de este sistema es un consultante que solicita patrocinio gratuito en el Patrocinio Jurídico para un caso particular.

\begin{table}[H]
\centering
\label{tab:client}
\begin{tabular}{|p{3cm}|p{2.5cm}|p{4.5cm}|p{4cm}|p{1cm}|}
\hline
\textbf{Atributo} & \textbf{Tipo de dato}  & \textbf{Descripción} & \textbf{Comentarios} & \textbf{No nulo} \\ \hline
id & int4 & Clave Primaria. & Incremental automático. & V \\ \hline
first\_name & varchar(150) & Nombre. &  & V \\ \hline
last\_name & varchar(150) & Apellido. &  & V \\ \hline
id\_type & varchar(10) & Tipo de documento. & Valores posibles: DOCUMENT y PASSPORT. & V \\ \hline
id\_value & varchar(20) & Valor del documento (Pasaporte o Número de documento). & & V \\ \hline
sex & varchar(6) & Sexo. & Valores posibles: MALE y FEMALE. & V \\ \hline
birth\_date & date & Fecha de nacimiento. & & V \\ \hline
address & text & Dirección donde vive. & & V \\ \hline
postal & int4 & Código postal. & & V \\ \hline
locality\_id & int4 & PK de localidad donde vive. & & V \\ \hline
marital\_status & varchar(10) & Estado civil. & Valores posibles: SINGLE, MARRIED, DIVORCED y WIDOWER. & V \\ \hline
housing\_type & varchar(20) & Tipo de vivienda. & Valores posibles: HOUSE, DEPARTMENT, TRAILER, STREET\_SITUATION. & V \\ \hline
studies & varchar(30) & Estudios alcanzados. & & V \\ \hline
email & varchar(254) & Correo electrónico. & & V \\ \hline
partner\_salary & int4 & Salario de la pareja del cliente. &  & V \\ \hline
employment & varchar(45) & Tabajo del cliente. &  & V \\ \hline
salary & int4 & Salario del cliente. &  & V \\ \hline
amount\_retirement & int4 & Monto de ingreso por jubilación. &  & V \\ \hline
amount\_pension & int4 & Monto de ingreso por pensión. &  & V \\ \hline
other\_income & int4 & varchar(45) &  & V \\ \hline
amount\_other\_income & int4 & Monto de otros ingresos del cliente. &  & V \\ 
\end{tabular}
\caption{Tabla client}
\end{table}


\section{Tabla Tel}\label{sec:table-tel}
La tabla \texttt{Tel} almacena los números de teléfono de un cliente.

\begin{table}[H]
\centering
\label{tab:family}
\begin{tabular}{|p{3cm}|p{2.5cm}|p{4.5cm}|p{4cm}|p{1cm}|}
\hline
\textbf{Atributo} & \textbf{Tipo de dato}  & \textbf{Descripción} & \textbf{Comentarios} & \textbf{No nulo} \\ \hline
id & int4 & Clave Primaria. & Incremental automático. & V \\ \hline
phone\_number & int4 & Número de teléfono. &  & V \\ \hline
id\_client & int4 & PK del cliente. &  & V \\ \hline
\end{tabular}
\caption{Tabla Family}
\end{table}



\section{Tabla Comment}\label{sec:table-comment}
La tabla \texttt{Comment} almacena los comentarios relacionados con las consultas realizadas. Se utiliza para realizar seguimientos de los casos (consultas). Estos son realizados por los profesores encargados de llevar a cabo dicho caso o consulta.

\begin{table}[H]
\centering
\label{tab:comment}
\begin{tabular}{|p{3cm}|p{2.5cm}|p{4.5cm}|p{4cm}|p{1cm}|}
\hline
\textbf{Atributo} & \textbf{Tipo de dato}  & \textbf{Descripción} & \textbf{Comentarios} & \textbf{No nulo} \\ \hline
id & int4 & Clave Primaria. & Incremental automático. & V \\ \hline
time\_stamp & timestamptz & Momento en que se creó el comentario. &  & V \\ \hline
text & text & Texto del comentario. &  & V \\ \hline
consultation\_id & int4 & PK de la consulta. &  & V \\ \hline
user\_id & int4 & PK del usuario que realizó la consulta. &  & V \\ \hline
\end{tabular}
\caption{Tabla Comment}
\end{table}


\section{Tabla File}\label{sec:table-file}
La tabla \texttt{File} almacena la información de los archivos adjuntos en los comentarios de una consulta o caso.

\begin{table}[H]
\centering
\label{tab:file}
\begin{tabular}{|p{3cm}|p{2.5cm}|p{4.5cm}|p{4cm}|p{1cm}|}
\hline
\textbf{Atributo} & \textbf{Tipo de dato}  & \textbf{Descripción} & \textbf{Comentarios} & \textbf{No nulo} \\ \hline
id & int4 & Clave Primaria. & Incremental automático. & V \\ \hline
time\_stamp & timestamptz & Momento en que se subió el archivo. &  & V \\ \hline
filename & varchar(255) & Nombre del archivo adjunto. &  & V \\ \hline
comment\_id & int4 & PK del comentario al que se adjuntó el archivo. &  & V \\ \hline
\end{tabular}
\caption{Tabla File}
\end{table}


\section{Tabla Consultation}\label{sec:table-consultation}
Una consulta es realizada por un cliente que solicita que un caso sea patrocinado por el sistema de patrocinio.

\begin{table}[H]
\centering
\label{tab:consultation}
\begin{tabular}{|p{3cm}|p{2.5cm}|p{4.5cm}|p{4cm}|p{1cm}|}
\hline
\textbf{Atributo} & \textbf{Tipo de dato}  & \textbf{Descripción} & \textbf{Comentarios} & \textbf{No nulo} \\ \hline
id & int4 & Clave Primaria. & Incremental automático. & V \\ \hline
tag & varchar(30) & Título o código de identificación de la consulta. & & V \\ \hline
client\_id & int4 & PK del cliente que realizó la consulta. & & V \\ \hline
availability\_state & varchar(12) & Estado de disponibilidad. & Valores posibles: CREATED, PENDING, ASSIGNED, REJECTED y ARCHIVED. & V \\ \hline
progress\_state & varchar(12)  & Estado de progreso. & Valores posibles: TODO, IN\_PROGRESS, DONE, PAUSED, BLOCKED, INCOMPLETE. & V \\ \hline
time\_stamp & timestamptz & Fecha de creación de la consulta. &  & V \\ \hline
start\_stamp & timestamptz & Fecha en que se comenzó a procesar el caso. &  & F \\ \hline
description & text & Descripción detallada de la consulta realizada por el cliente. &  & V \\ \hline
opponent & varchar(500) & Oponente de la consulta. &  Parte contraria a quien presenta la demanda, puede ser una persona física, una empresa, una organización, una entidad gubernamental, etc. & V \\ \hline

\end{tabular}
\caption{Tabla Consultation}
\end{table}



\section{Tabla Request Consultation}\label{sec:table-req-consultation}
La tabla \texttt{Request Consultation} se utiliza para registrar solicitudes de consulta dirigidas a una comisión, con el fin de que esta se haga cargo del caso o consulta correspondiente.

\begin{table}[H]
\centering
\label{tab:req-consult}
\begin{tabular}{|p{3cm}|p{2.5cm}|p{4.5cm}|p{4cm}|p{1cm}|}
\hline
\textbf{Atributo} & \textbf{Tipo de dato}  & \textbf{Descripción} & \textbf{Comentarios} & \textbf{No nulo} \\ \hline
consultation\_id & int4 & Clave Primaria y PK de la consulta asociada. & & V \\ \hline
destiny\_board\_id & int4 &  PK del tablero de la comisión destino a la que se envía la solicitud. &  & V \\ \hline
state & varchar(12) & Estado de la solicitud. & Valores posibles: PENDING, ACCEPTED, REJECTED. Por defecto es PENDING. & V \\ \hline
resolution\_time stamp & timestamptz & Tiempo de resolución de la solicitud, es decir, el momento en que se rechaza o acepta. & Valor por defecto es NULL. & F \\ \hline
\end{tabular}
\caption{Tabla Request Consultation}
\end{table}

Para garantizar que no haya más de una solicitud pendiente para una consulta, se estableció como única la combinación de los campos 'consultation', 'state' y 'resolution\_timestamp'.


\section{Tabla Panel}\label{sec:table-panel}
Un panel es un espacio dentro de un tablero donde se agrupan diferentes tarjetas o tickets de consultas. Cada tablero puede tener más de un panel, y estos sirven para organizar las consultas por profesor, grupo, estado u otros criterios.

\begin{table}[H]
\centering
\label{tab:panel}
\begin{tabular}{|p{3cm}|p{2.5cm}|p{4.5cm}|p{4cm}|p{1cm}|}
\hline
\textbf{Atributo} & \textbf{Tipo de dato}  & \textbf{Descripción} & \textbf{Comentarios} & \textbf{No nulo} \\ \hline
id & int4 & Clave Primaria. & Incremental automático. & V \\ \hline
title & varchar(255) &  Título del panel. &  & V \\ \hline
board\_id & int4 & PK del tablero al que pertenece el panel. &  & V \\ \hline
\end{tabular}
\caption{Tabla Panel}
\end{table}