\chapter{Conclusiones}\label{cap:conclusiones}

Durante el desarrollo de este proyecto, me enfrenté a diversos desafíos que impulsaron la expansión de mis conocimientos en diferentes frameworks y lenguajes de programación. La inmersión en el paradigma de programación funcional al trabajar con React proporcionó una perspectiva declarativa que enriqueció mi capacidad para abordar problemas de manera diferente.

Este año no solo significó la adquisición de nuevas habilidades técnicas, sino también el perfeccionamiento de buenas prácticas y la exploración de nuevas tecnologías, así como la gestión de proyectos y la relación con el cliente. La experiencia en el desarrollo del Sistema de Gestión de Casos no solo amplió mi comprensión teórica, sino que también me permitió aplicar esos conocimientos participando en todas las fases del proyecto.

El desarrollo e implementación del Sistema de Gestión de Casos (Case Management System) representa un hito significativo en el ámbito del patrocinio jurídico, proporcionando una solución tecnológica adaptable a las necesidades específicas del laboratorio IALAB de la Universidad de Buenos Aires. A lo largo de este proceso, se han alcanzado varios objetivos clave que abarcan desde la automatización de la captura de consultas hasta la gestión eficiente de los casos y su asignación a comisiones especializadas.

En primer lugar, la integración con Google Forms estableció un canal eficaz para la recepción de consultas. La aplicación de estrategias y scripts personalizados en Google Forms superó desafíos como la limitación de tipos de datos, facilitando la gestión de preguntas interdependientes y proporcionando una interfaz intuitiva para la gestión de credenciales.

La arquitectura modular y la implementación de tecnologías como Django, React y Docker impulsaron la escalabilidad y el mantenimiento del sistema. La adopción de buenas prácticas en el desarrollo, como la estructuración modular del backend y frontend, el uso de contenedores para la implementación y la aplicación de patrones de diseño como el Pub/Sub, contribuyó a la solidez y flexibilidad del sistema.

La creación de un tablero de trabajo para la consultoría resultó instrumental para la toma de decisiones y el seguimiento de asignaciones. La capacidad de arrastrar y soltar consultas, la visualización detallada del historial de asignaciones y los comentarios y calendarios de las consultas optimizaron la eficiencia, trazabilidad y el control en la gestión de casos.

En cuanto a la seguridad, la implementación de roles específicos y la creación de un mecanismo de autorización fortalecieron la protección de datos y la administración de credenciales, asegurando la confidencialidad e integridad de la información sensible.

El sistema sienta las bases para un sistema modular que ofrece flexibilidad para un proceso continuo que seguirá en desarrollo. En futuras versiones, se explorarán oportunidades de mejora, como fortalecer la integración con Google Forms para permitir la edición de datos de clientes y consultas después de su envío inicial, realización de informes y estadísticas, como la clasificación de los casos por categorías, y la integración de filtros automáticos para asistir a los tomadores de casos en el Patrocinio Jurídico de la Universidad de Buenos Aires.