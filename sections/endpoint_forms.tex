\chapter{Endpoints de Formularios}\label{cap:anexo-expoint-forms}

\section{Endpoint del Formulario Registro de Cliente}

\subsection*{URL}
\texttt{http://\{\{ip\}\}:\{\{port\}\}/api/clients/client/form/}

\subsection*{Método}
POST

\subsection*{Body de Ejemplo:}
A continuación se presenta un ejemplo.
\begin{lstlisting}[caption=Body de Ejemplo, label=example-body]
{
    "first_name": "Martina",
    "last_name": "Gutierrez",
    "id_type": "DOCUMENT",
    "id_value": "42304328",
    "birth_date": "2000-11-13",
    "sex": "FEMALE",
    "marital_status": "SINGLE",
    "studies": "INCOMPLETE_UNIVERSITY",
    "email": "martina2000@gmail.com",
    "housing_type": "HOUSE",
    "locality": 9,
    "address": "Av. Patria 1234",
    "postal": "5000",
    "employment": "programmer",
    "salary": "123",
    "other_income": "No tengo otros ingresos",
    "amount_other_income": "0",
    "amount_retirement": "0",
    "amount_pension": "0",
    "vehicle": "No tengo otros vehiculos",
    "tel": [
        "3512254210",
        "+54 9 25578784"
    ],
    "partner_salary": "0"
}
\end{lstlisting}
\subsection*{Campos del Body:}

\begin{table}[H]
    \centering
    \begin{tabular}{|l|l|p{5cm}|p{5cm}|}
        \hline
        \textbf{Campo} & \textbf{Tipo} & \textbf{Opciones} & \textbf{Descripción} \\ \hline
        first\_name & String & & Nombre del cliente. \\ \hline
        last\_name & String & & Apellido del cliente. \\ \hline
        id\_type & String & DOCUMENT, PASSPORT & Tipo de documento de identidad. \\ \hline
        id\_value & String & & Valor del documento de identidad. \\ \hline
        birth\_date & String & & Fecha de nacimiento del cliente. \\ \hline
        sex & String & MALE, FEMALE & Género del cliente. \\ \hline
        marital\_status & String & SINGLE, MARRIED, DIVORCED, WIDOWER & Estado civil del cliente. \\ \hline
        studies & String & INCOMPLETE\_PRIMARY, COMPLETE\_PRIMARY, INCOMPLETE\_SECONDARY, COMPLETE\_SECONDARY, INCOMPLETE\_TERTIARY, COMPLETE\_TERTIARY, INCOMPLETE\_UNIVERSITY, COMPLETE\_UNIVERSITY & Nivel de estudios del cliente. \\ \hline
        email & String & & Correo electrónico del cliente. \\ \hline
        housing\_type & String & HOUSE, DEPARTMENT, TRAILER, STREET\_SITUATION & Tipo de vivienda del cliente. \\ \hline
        locality & Integer & & ID de localidad del cliente. \\ \hline
        address & String & & Dirección del cliente. \\ \hline
        postal & String & & Código postal del cliente. \\ \hline
        employment & String & & Ocupación del cliente. \\ \hline
        salary & String & & Salario del cliente. \\ \hline
        other\_income & String & & Otros ingresos del cliente. \\ \hline
        amount\_other\_income & String & & Monto de otros ingresos. \\ \hline
        amount\_retirement & String & & Monto de jubilación. \\ \hline
        amount\_pension & String & & Monto de pensión. \\ \hline
        vehicle & String & & Información sobre vehículos del cliente. \\ \hline
        tel & List & & Lista de números de teléfono del cliente. \\ \hline
        partner\_salary & String & & Salario del cónyuge (si aplica). \\ \hline
    \end{tabular}
    \caption{Descripción de campos del Body}
    \label{tab:body-fields-client}
\end{table}


\subsection*{Headers:}

\begin{itemize}
    \item \textbf{Authorization:} Token \{\{TOKEN\}\}
    \item \textbf{Content-Type:} application/json
\end{itemize}






\section{Endpoint del Formulario Registro de Hijo}

\subsection*{URL}
\texttt{http://\{\{ip\}\}:\{\{port\}\}/api/clients/son/form/}

\subsection*{Método}
POST

\subsection*{Campos del Body:}

\begin{table}[H]
    \centering
    \begin{tabular}{|l|l|l|p{7cm}|}
        \hline
        \textbf{Campo} & \textbf{Tipo} & \textbf{Opciones} & \textbf{Descripción} \\ \hline
        id\_consultant & String & & Identificación del consultor asociado. \\ \hline
        first\_name & String & & Nombre del hijo. \\ \hline
        last\_name & String & & Apellido del hijo. \\ \hline
        id\_type & String & DOCUMENT, PASSPORT & Tipo de documento de identidad del hijo. \\ \hline
        id\_value & String & & Valor del documento de identidad del hijo. \\ \hline
        birth\_date & String & & Fecha de nacimiento del hijo. \\ \hline
        sex & String & MALE, FEMALE & Género del hijo. \\ \hline
        locality & Integer & & ID de la localidad del hijo. \\ \hline
        address & String & & Dirección del hijo. \\ \hline
    \end{tabular}
    \caption{Descripción de campos del Body}
    \label{tab:body-fields-son}
\end{table}

\newpage
\subsection*{Body de Ejemplo:}

\begin{lstlisting}[caption=Body de Ejemplo, label=example-body-son]
{
    "id_consultant": "42304328",
    "first_name": "Matilda",
    "last_name": "Gonzales",
    "id_type": "PASSPORT",
    "id_value": "123456789",
    "birth_date": "2020-11-14",
    "sex": "FEMALE",
    "locality": 9,
    "address": "Av. Patria 1234"
}
\end{lstlisting}


\subsection*{Headers:}

\begin{itemize}
    \item \textbf{Authorization:} Token \{\{TOKEN\}\}
    \item \textbf{Content-Type:} application/json
\end{itemize}




\section{Endpoint del Formulario Consulta}

\subsection*{URL}
\texttt{http://\{\{ip\}\}:\{\{port\}\}/api/consultations/consultation/form/}

\subsection*{Método}
POST

\subsection*{Body de Ejemplo:}

\begin{lstlisting}[caption=Body de Ejemplo, label=example-body-consultation]
{
    "client": "42304328",
    "tag": "Asesoramiento Legal",
    "description": "Necesito asesoramiento legal con respecto a una disputa contractual con la Corporacion XYZ. El contrato involucra la venta de bienes y hay desacuerdos sobre los plazos de entrega y los terminos de pago.",
    "opponent": "Corporacion XYZ"
}
\end{lstlisting}

\subsection*{Campos del Body:}

\begin{table}[H]
    \centering
    \begin{tabular}{|l|l|l|p{7cm}|}
        \hline
        \textbf{Campo} & \textbf{Tipo} & \textbf{Opciones} & \textbf{Descripción} \\ \hline
        client & String & & Identificación del cliente asociado a la consulta. \\ \hline
        tag & String & & Etiqueta de la consulta. \\ \hline
        description & String & & Descripción detallada de la consulta. \\ \hline
        opponent & String & & Oponente o entidad involucrada en la consulta. \\ \hline
    \end{tabular}
    \caption{Descripción de campos del Body}
    \label{tab:body-fields-consultation}
\end{table}

\subsection*{Headers:}

\begin{itemize}
    \item \textbf{Authorization:} Token \{\{TOKEN\}\}
    \item \textbf{Content-Type:} application/json
\end{itemize}
