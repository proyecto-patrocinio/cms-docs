\chapter{Introducción}
\label{cap:introduccion}

\section{Introducción}
\label{sec:introduccion:intro}
En este capítulo se delimita el contexto de la tesis, al mismo tiempo que se describen los objetivos, el alcance y la estructura de este documento.

\section{Contexto del PI}
\label{sec:introduccion:contexto}
El origen de este producto surgió a raíz del contacto con el laboratorio IALAB, quienes solicitaron el desarrollo de un sistema de gestión diseñado específicamente para potenciar y optimizar el Patrocinio Jurídico de la Facultad de Derecho de la Universidad de Buenos Aires.

El ``Patrocinio Jurídico Gratuito'', con casi un siglo de dedicación, se erige como un proyecto comprometido con la provisión de asistencia legal a aquellos en situación de vulnerabilidad económica y social. Esta iniciativa, integrada como práctica profesional en el plan de estudios de la carrera de abogacía, se distingue como un servicio singular en el país.

La problemática identificada en la administración de casos por parte de esta entidad se centra en la utilización actual de hojas de cálculo y herramientas de pago empleadas para la gestión de casos, y Google Forms como sistema de recopilación de datos. Sin embargo, estas soluciones no satisfacen completamente las necesidades específicas del organismo, que requiere un software integral para la gestión, organización y automatización parcial de sus procesos.

En esta fase del proyecto, se busca reemplazar y unificar el sistema de gestión de casos, como así también automatizar eficientemente la carga de formularios en el sistema, contribuyendo un producto de valor al Proyecto Patrocinio.

El desarrollo de este sistema representa la concepción de una primera base que servirá como cimiento para futuras expansiones. Este enfoque busca establecer una plataforma modular y versátil que aborde las necesidades actuales del patrocinio jurídico de la UBA.



\section {Objetivo General del Proyecto}
\label{sec:objetivos:generales}
Diseñar e implementar un software integral de gestión de casos para el patrocinio jurídico de la Universidad de Buenos Aires (UBA).

\section {Objetivos Específicos del Proyecto}
\label{sec:objetivos:especificos}

\begin {enumerate}
\item Realizar un análisis exhaustivo para comprender a fondo las limitaciones al proceso actualmente en uso para el procesamiento de los casos, así como al sistema de recolección de datos basado en Google Forms.
\item Relevar, analizar y construir los requisitos tanto funcionales como no-funcionales solicitados por el cliente.
\item Diseñar e implementar una arquitectura modular que permita una gestión eficiente y escalable de casos.
\item Diseñar una estructura de base de datos que permita una gestión eficaz de la información.
\end{enumerate}

\section {Alcance del Proyecto}
\label{sec:alcance}
En el marco de este proyecto, se abordarán los siguientes procesos:

\begin {enumerate}
\item Diseño, Desarrollo e Implementación del Software de Gestión de Casos:

El enfoque principal estará en la concepción, desarrollo y despliegue de un software de gestión de casos diseñado específicamente para satisfacer las necesidades del patrocinio jurídico de la Universidad de Buenos Aires (UBA).

\item Integración con Google Forms:

Se incorporará la integración de Google Forms para automatizar la carga eficiente de formularios en el sistema, mejorando así la recopilación de datos y simplificando el proceso para el patrocinio jurídico.

\item Optimización del Sistema de Solicitud y Control Histórico:

El software se orientará hacia la facilitación y optimización del sistema de solicitud, asegurando un control histórico efectivo de los casos por comisión. Esto se traducirá en una gestión más eficiente de los procesos asociados con la presentación y seguimiento de casos.

\item Registro Histórico por Comisión:

Se contemplará la mejora del registro histórico mediante la posibilidad de cargar archivos relevantes y comentarios asociados a cada caso. Esto permitirá un seguimiento más detallado y completo de la evolución de los casos a lo largo del tiempo.
\end{enumerate}

\section{Acrónimos y Abreviaturas}
\label{sec:acronimos}

En el presente documento, se utilizan los siguientes acrónimos y abreviaturas:

\begin{table}[H]
    \centering
    \begin{tabular}{|c|p{10cm}|}
    \hline
         \textbf{Acrónimos} & \textbf{Descripción}\\
    \hline
         API & Interfaz de Programación de Aplicaciones (por sus siglas en inglés, \textit{Application Programming Interface})\\
    \hline
         HTTP & Protocolo de Transferencia de Hipertexto (por sus siglas en inglés, \textit{Hypertext Transfer Protocol})\\
    \hline
         IP & Protocolo de Internet (\textit{Internet Protocol})\\
    \hline
         JSON & Notación de Objetos de JavaScript (\textit{JavaScript Object Notation}) \\
    \hline
         REST & Transferencia de Estado Representacional (\textit{Representational State Transfer}) \\
    \hline
        DNS & Sistema de Nombres de Dominio (\textit{Domain Name System}) \\
    \hline
        SSL/TLS & Capa de Conexión Segura / Protocolo de Seguridad de la Capa de Transporte (\textit{Secure Sockets Layer / Transport Layer Security}) \\
    \hline
        SQL & Lenguaje de Consulta Estructurada (\textit{Structured Query Language}) \\
    \hline
        CORS & Intercambio de recursos de origen cruzado (\textit{Cross Origin Resource Sharing})\\
    \hline
         CSRF & Falsificación de Petición en Sitios Cruzados (\textit{Cross-Site Request Forgery}) \\
    \hline
         NIST & Instituto Nacional de Estándares y Tecnología (\textit{National Institute of Standards and Technology}) \\
    \hline
        CI & Integración Continua (\textit{Continuous Integration}) \\
    \hline
        CD & Entrega Continua (\textit{Continuous Delivery}) \\
    \hline
        UBA & Universidad de Buenos Aires \\
    \hline
        UNC & Universidad Nacional de Córdoba \\
    \hline
       ASGI & Interfaz de Puerta de Enlace Asíncrona del Servidor (\textit{Asynchronous Server Gateway Interface}) \\
    \hline
        WSGI & Interfaz de Puerta de Enlace del Servidor Web (\textit{Web Server Gateway Interface}) \\
    \hline
        
    \end{tabular}
    \caption{Lista de Acrónimos y Abreviaturas}
    \label{tab:my_label}
\end{table}
