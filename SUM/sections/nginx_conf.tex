\chapter{Configuración de Nginx}\label{cap:apendix-nginx}
% Configuración de estilo para el código
\lstset{
    language=sh,
    backgroundcolor=\color{white},
    basicstyle=\footnotesize\ttfamily,
    keywordstyle=\color{blue},
    commentstyle=\color{green},
    stringstyle=\color{red},
    captionpos=b,
    breaklines=true,
    showstringspaces=false,
    numbers=left,
    numberstyle=\tiny,
    frame=single,
    rulecolor=\color{black}
}

\section{Configuración del Ngnix como Reverse Proxy}\label{subsec:nginx_explicacion}
El archivo de configuración Nginx utilizado como proxy reverse en el Stack de contenedores es el siguiente:
\begin{lstlisting}[caption={Configuración de Nginx Reverse Proxy}, label={cod:nginx}, captionpos=b]
limit_conn_zone $binary_remote_addr zone=addr:10m;

upstream gunicorn{
    server appserver:8000;
}

upstream daphne{
    server appserver:8001;
}

upstream frontend{
    server frontend:3000;
}

server {
    client_body_timeout 5s;
    client_header_timeout 5s; # Closing Slow Connections
    server_tokens off;
    listen 80;
    listen  [::]:80;
    server_name  proyecto-patrocinio.fcefyn.unc.edu.ar;

    location / {
        limit_req zone=mylimit burst=20 nodelay;
        limit_conn addr 10;
        try_files $uri @proxy_frontend;
    }
    location /api {
        limit_req zone=mylimit burst=20 nodelay;
        limit_conn addr 10;
        try_files $uri @proxy_api_wsgi;
    }
    location /ws {
        limit_req zone=mylimit burst=20 nodelay;
        limit_conn addr 10;
        try_files $uri @proxy_api_asgi;
    }
    location /admin {
        limit_req zone=mylimit burst=20 nodelay;
        limit_conn addr 10;
        root /usr/src/app/django_static;
        include /etc/nginx/mime.types;
        try_files $uri @proxy_api_wsgi;
    }

    # redirect to django app: React server
    location @proxy_frontend {
        proxy_pass http://frontend;
        proxy_redirect off;
        proxy_cache_bypass  $http_upgrade;
        proxy_set_header Upgrade           $http_upgrade;
        proxy_set_header Connection        "upgrade";
        proxy_set_header Host              $host;
        proxy_set_header X-Real-IP         $remote_addr;
        proxy_set_header X-Forwarded-For   $proxy_add_x_forwarded_for;
        proxy_set_header X-Forwarded-Proto $scheme;
        proxy_set_header X-Forwarded-Host  $host;
        proxy_set_header X-Forwarded-Port  $server_port;
        proxy_set_header Cookie $http_cookie;
    }

    # redirect to django app: API REST
    location @proxy_api_wsgi {
        proxy_pass http://gunicorn;
        proxy_redirect off;
        proxy_cache_bypass  $http_upgrade;
        proxy_set_header Upgrade           $http_upgrade;
        proxy_set_header Connection        "upgrade";
        proxy_set_header Host              $host;
        proxy_set_header X-Real-IP         $remote_addr;
        proxy_set_header X-Forwarded-For   $proxy_add_x_forwarded_for;
        proxy_set_header X-Forwarded-Proto $scheme;
        proxy_set_header X-Forwarded-Host  $host;
        proxy_set_header X-Forwarded-Port  $server_port;
        proxy_set_header Cookie $http_cookie;
    }

    # redirect to django app with WebSocket
    location @proxy_api_asgi {
        proxy_pass http://daphne;
        proxy_redirect off;
        proxy_cache_bypass  $http_upgrade;
        proxy_set_header Upgrade           $http_upgrade;
        proxy_set_header Connection        "upgrade";
        proxy_set_header Host              $host;
        proxy_set_header X-Real-IP         $remote_addr;
        proxy_set_header X-Forwarded-For   $proxy_add_x_forwarded_for;
        proxy_set_header X-Forwarded-Proto $scheme;
        proxy_set_header X-Forwarded-Host  $host;
        proxy_set_header X-Forwarded-Port  $server_port;
        proxy_set_header Cookie $http_cookie;
    }

    location /django_static/ {
        limit_req zone=mylimit burst=20 nodelay;
        limit_conn addr 10;
        autoindex on;
        alias /usr/share/nginx/staticfiles/cms/django/;
    }
}
\end{lstlisting}







\section{Configuración del Ngnix como Servidor}

Por otro lado, el servidor contaba con un Servidor Nginx instalado y en funcionamiento sirviendo otras aplicaciones.
Para agregar el servicio Case Managment System se agregó el siguiente archivo en el directorio \textbf{etc/nginx/conf.d/proyecto-patrocinio.fcefyn.unc.edu.ar.conf}:



\begin{lstlisting}[caption={Configuración de Nginx en el Servidor}, label={cod:nginx}, captionpos=b]

server {
    server_name proyecto-patrocinio.fcefyn.unc.edu.ar;

    location / {
        proxy_pass http://localhost:8081;
        proxy_redirect off;
        proxy_cache_bypass  $http_upgrade;
        proxy_set_header Upgrade           $http_upgrade;
        proxy_set_header Connection        "upgrade";
        proxy_set_header Host              $host;
        proxy_set_header X-Real-IP         $remote_addr;
        proxy_set_header X-Forwarded-For   $proxy_add_x_forwarded_for;
        proxy_set_header X-Forwarded-Proto $scheme;
        proxy_set_header X-Forwarded-Host  $host;
        proxy_set_header X-Forwarded-Port  $server_port;
        proxy_set_header Cookie $http_cookie;
    }

    listen 443 ssl; # managed by Certbot
    ssl_certificate /etc/letsencrypt/live/proyecto-patrocinio.fcefyn.unc.edu.ar/fullchain.pem; # managed by Certbot
    ssl_certificate_key /etc/letsencrypt/live/proyecto-patrocinio.fcefyn.unc.edu.ar/privkey.pem; # managed by Certbot
    include /etc/letsencrypt/options-ssl-nginx.conf; # managed by Certbot
    ssl_dhparam /etc/letsencrypt/ssl-dhparams.pem; # managed by Certbot

}

server {
    if ($host = proyecto-patrocinio.fcefyn.unc.edu.ar) {
        return 301 https://$host$request_uri;
    } # managed by Certbot


    listen 80;
    server_name proyecto-patrocinio.fcefyn.unc.edu.ar;
    return 404; # managed by Certbot


}
\end{lstlisting}